%begin-include

\unchapter{Quasi-preface}
\begin{para*}[Leaving the cave]
This book is the book I would have liked to read when I began my journey as a mathematics undergraduate student.
It aims to provide a ``baptism of fire'' to the world of mathematics by first diving fairly deep into the wonders of mathematical logic and then building some basic mathematics on top of that. Do not be fooled by the seemingly innocent section titles; this is not your standard introductory book.
With the exception of the preliminaries (which are fully informal), I have tried to do everything with as much rigour and formal correctness as possible. 
\end{para*}


\begin{para*}[How this book is organised]
This book is divided into chapters which are divided into sections which are divided into blocks. These blocks are numbered within the sections, so the first block of section 2 of chapter $n$ will be labelled as 2.1.
References to blocks in the same chapter will use their label. Blocks in other chapters (e.g., block 3.4 in chapter I), will be referenced writing the block label after the chapter number (as in I-3.4).

At the end of each chapter, there are some exercises meant for you to work on the material and develop your skills. These exercises are ordered in increasing order of difficulty.
\end{para*}

\begin{para*}[This book is a work in progress]
The writing of this book is, and will always be, an ongoing effort.
Nothing human is perfect and, therefore, nothing human is ever truly finished. I thus believe that, using the power of the internet to allow for continuous change, I should always be open to the possibility of enlarging and improving this book.

On future releases of this book, you should expect to see both many corrections on what has already been written and lots of new material. I have plans to include chapters on arithmetic and euclidean geometry. 

While you are reading, please keep in mind that you may find some typos and errors. I have done my best to catch as many of them as I could, but, you know, I am human.
\end{para*}

\begin{para*}[I want to hear from you]
This book should be accessible to \emph{anyone} who wants to read it. If you ever feel stuck with the material, please, let me know. I will be more than happy to answer any questions you may have. Moreover, your questions will help me understand which parts of the book need more attention.

I make myself no illusion. I know there is still a lot of work to be done and a big margin for improvement. Any feedback --- whether positive or negative --- will always be welcome. Any. Please, do not hesitate to get in touch.  
\end{para*}

\begin{para*}[Why is this a quasi-preface?]
Prefaces are often home to acknowledgements. I certainly have people to thank for their support and help in writing this book. Nonetheless, I would rather wait until this project evolves a little bit more and is worthy of including the names of these people.
\end{para*}
