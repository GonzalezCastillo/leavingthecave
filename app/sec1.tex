%begin-include

\section{Additional results in formal logic}

\begin{lemma}
Let $A$ and $B$ be any propositional forms. The formulas
\begin{statements}
\item \label{pcl1} $\lnot A \limplies (A \limplies B)$,
\item \label{pcl2} $A \limplies \lnot\lnot A$,
\item \label{pcl3} $A \limplies (\lnot B \limplies \lnot (A\limplies B))$,
\item \label{pcl4} $(A\limplies B) \limplies ( (\lnot A \limplies B) \limplies B)$,
\end{statements}
are theorems of the formal system $\mathsf{P}$ of propositional logic.
\label{pcl}
\end{lemma}

\begin{proof}
What follows are some tedious, long and boring mechanical proofs.
It is hard to say that you will gain anything from carefully reading them other than convincing yourself that this lemma is true.
You have been advised. Proceed at your own discretion.
\begin{parlist}
\item It is immediate that $A, (A \limplies B), (B \limplies C) \vdash C$, so a simple application of the deduction theorem \ref[fund]{dedthmprop} reveals that
\[(A\limplies B), (B\limplies C) \vdash (A \limplies C).\]
This is known as the \emph{hypothetical syllogism rule} (HS). 

With HS in our toolbox, we can perform the following proof in the formal system $\mathsf{P}$.
\begin{deduction}{notaaimpb}
\dstep[1]{\ref[fund]{axp1}}{\lnot A \limplies (\lnot B \limplies \lnot A),}
\dstep[2]{\ref[fund]{axp3}}{(\lnot B \limplies \lnot A) \limplies (A \limplies B),}
\dstep{HS on \dref{1} and \dref{2}}{\lnot A \limplies (A \limplies B).}
\end{deduction}

\item No fancy artefacts are required for this proof. We just need some patience.

\begin{deduction}{browerdisapproves}
\dstep[1]{\ref{pcl1}}{\lnot\lnot A \limplies (\lnot A \limplies \lnot\lnot\lnot A),}
\dstep[2]{\ref[fund]{axp3}}{(\lnot A \limplies \lnot\lnot\lnot A) \limplies (\lnot\lnot A \limplies A),}
\dstep[3]{HS on \dref{1}, \dref{2}}{\lnot\lnot A \limplies (\lnot \lnot A \limplies A),}
\dstep[4]{\ref[fund]{axp2}}{(\lnot\lnot A \limplies (\lnot\lnot A \limplies A)) \limplies ( (\lnot \lnot A \limplies \lnot \lnot A) \limplies (\lnot\lnot A \limplies A)),}
\dstep[5]{MP on \dref{3}, \dref{4}}{(\lnot \lnot A \limplies \lnot \lnot A) \limplies (\lnot\lnot A \limplies A),}
\dstep[6]{\ref[fund]{dedaa}}{\lnot\lnot A \limplies \lnot\lnot A,}
\dstep[7]{MP on \dref{5}, \dref{6}}{\lnot\lnot A \limplies A,}
\dstep[8]{Instance of \dref{7}}{\lnot\lnot\lnot A \limplies \lnot A,}
\dstep[9]{\ref[fund]{axp3}}{(\lnot\lnot\lnot A \limplies \lnot A) \limplies (A \limplies \lnot\lnot A),}
\dstep{MP on \dref{8}, \dref{9}}{A \limplies \lnot\lnot A.}
\end{deduction}

\item For this proof, we need some background results.
Firstly,we should notice that, for any formulas $A$ and $B$, we have $\lnot\lnot A, (A\limplies B) \vdash \lnot\lnot B$.
This follows from some simple applications of modus ponens taking into consideration that, as we showed in the previous deduction, $\lnot\lnot A \limplies A$ and $B \limplies \lnot\lnot B$ are both theorems of $\mathsf{P}$.
In addition, the application of the deduction theorem on this result yields the existence of a proof for
\begin{equation}
(A\limplies B) \limplies (\lnot\lnot A \limplies B).
\tag{$1*$}
\label{prelp31}
\end{equation}
Furthermore, some other applications of the deduction of theorem and the modus ponens rule on $A,(A\limplies B) \vdash B$ reveal that
\begin{equation}
A \limplies ( (A \limplies B) \limplies B)
\tag{$2*$}
\label{prelp32}
\end{equation}
is another theorem of $\mathsf{P}$.

With these preliminaries out of the way, we can now safely proceed to our proof.
\begin{deduction}{compleximplication}
\dstep[1]{\eqref{prelp31}}{(A \limplies B) \limplies (\lnot\lnot A \limplies \lnot\lnot B),}
\dstep[2]{\ref[fund]{axp3}}{(\lnot\lnot A \limplies \lnot\lnot B) \limplies (\lnot B \limplies \lnot A),}
\dstep[3]{MP on \dref{1}, \dref{2}}{(A\limplies B) \limplies (\lnot B \limplies \lnot A),}
\dstep[4]{Instance of \dref{3}}{( (A\limplies B) \limplies B) \limplies (\lnot B \limplies \lnot(A\limplies B)),}
\dstep[5]{\eqref{prelp32}}{A \limplies ( (A\limplies B) \limplies B),}
\dstep{HS on \dref{4}, \dref{5}}{A \limplies (\lnot B \limplies\lnot(A\limplies B)).} 
\end{deduction}
By the way, observe that we have also shown
\begin{equation}
(A \limplies B) \limplies (\lnot B \limplies \lnot A)
\tag{$3*$}
\label{axp3vv}
\end{equation}
to be a theorem of $\mathsf{P}$.

\item As an auxiliary result, we need to prove that, given any formula $A\in L_P$,  the propositional form
\begin{equation}
(\lnot A \limplies A) \limplies A
\tag{$1\star$}
\label{prelp41}
\end{equation}
is a theorem in the formal system $\mathsf{P}$. If we consider an arbitrary $B \in L_P$, this can be through from the following deduction.
\begin{deduction}{auxresult3}
\dstep[1]{\ref{pcl1}}{\lnot A \limplies (A \limplies \lnot B),}
\dstep[2]{\ref[fund]{axp2}}{(\lnot A \limplies (A \limplies \lnot B)) \limplies ( (\lnot A \limplies A) \limplies (\lnot A \limplies \lnot B)),}
\dstep[3]{MP on \dref{1}, \dref{2}}{ (\lnot A \limplies A) \limplies (\lnot A \limplies \lnot B),}
\dstep[4]{\ref[fund]{axp3}}{(\lnot A \limplies \lnot B) \limplies (B \limplies A),}
\dstep[5]{HS on \dref{3}, \dref{4}}{ (\lnot A \limplies A) \limplies (B \limplies A),}
\dstep[6]{Instance of \dref{5}}{(\lnot A \limplies A) \limplies ( (\lnot A \limplies A) \limplies A),}
\dstep[7]{\ref[fund]{axp2}}{( (\lnot A \limplies A) \limplies ( ( \lnot A \limplies A) \limplies A ) ) \limplies ( ( ( \lnot A \limplies A) \limplies (\lnot A \limplies A)) \limplies ( (\lnot A \limplies A) \limplies A)),}
\dstep[8]{MP on \dref{6}, \dref{7}}{( (\lnot A \limplies A) \limplies (\lnot A\limplies A)) \limplies ( (\lnot A \limplies A) \limplies A),}
\dstep[9]{\ref[fund]{dedaa}}{(\lnot A \limplies A) \limplies (\lnot A \limplies A),}
\dstep{MP on \dref{8}, \dref{9}}{(\lnot A \limplies A) \limplies A.}
\end{deduction}

In addition, it is immediate that $\lnot B, (\lnot B \limplies \lnot A), (\lnot A \limplies B) \vdash B$, which, after some applications of the deduction theorem, shows that
\begin{equation}
(\lnot B \limplies \lnot A) \limplies ( (\lnot A \limplies B) \limplies (\lnot B \limplies B)).
\tag{$2\star$}
\label{prelp42}
\end{equation}
is another theorem of propositional logic.

We will now proceed to deduce $B$ from $(A\limplies B)$ and $(\lnot A \limplies B)$. This will, through some applications of the deduction theorem, yield the result we wanted to prove.
\begin{deduction}{auxresult4}
\dstep[1]{Hypothesis}{A \limplies B,}
\dstep[2]{\ref{pcl3}\eqref{axp3vv}}{(A\limplies B)\limplies (\lnot B \limplies \lnot A),}
\dstep[3]{MP on \dref{1}, \dref{2}}{\lnot B \limplies \lnot A,}
\dstep[4]{\eqref{prelp42}}{(\lnot B \limplies \lnot A) \limplies ( (\lnot A \limplies B) \limplies (\lnot B \limplies B),}
\dstep[5]{MP on \dref{3}, \dref{4}}{(\lnot A \limplies B) \limplies (\lnot B \limplies B),}
\dstep[6]{Hypothesis}{\lnot A \limplies B,}
\dstep[7]{MP on \dref{5}, \dref{6}}{ \lnot B \limplies B,}
\dstep[8]{\eqref{prelp41}}{(\lnot B \limplies B) \limplies B,}
\dstep{MP on \dref{7}, \dref{8}}{B.}
\end{deduction}
Having shown that $(A \limplies B), (\lnot A \limplies B) \vdash B$, the deduction theorem leads us to $\vdash (A\limplies B) \limplies ( (\lnot A \limplies B) \limplies B)$. This concludes the proof.
\end{parlist}
\end{proof}

\begin{lemma}
Let $A\in L_P$ be any propositional and let $i$ be any interpretation of propositional logic together with its induced valuation function $v_i$.
Without loss of generality, let $x_1,\ldots,x_k$ be the only distinct propositional variables appearing in $A$.

For every propositional form $\phi \in L_{P}$, we define $\phi^i$ as $\phi$ if $v_i(\phi) = 1$ and $\phi^i$ as $\lnot \phi$ if, otherwise, $v_i(\phi) = 0$.
Under these conditions,
\[ x_1^i,\ldots,x_k^i \vdash A^i.\]
\label{<+label+>}
\end{lemma}

\begin{proof}
We will prove this by induction on the number of connectives in $A$.
If $A$ has no connectives, then it will simply be the variable $x_1$ and, clearly, $x_1^i \vdash x_1^i$, so the base case is obvious.

Let us assume the result to hold for any formulas with $n$ or less connectives and show it for an arbitrary formula $A$ with $n+1$.
Necessarily, $A$ will be of the form $\lnot X$ or $X \limplies Y$ for $X,Y \in F_P$.
In either case, the result will hold, by hypothesis, for the subformulas $X$ and $Y$.

If $A$ is of the form $\lnot X$, we may have $v_i(X) = 0$ or $v_i(X) = 1$. If $v_i(X) = 0$, then $X^i$ will be $\lnot X$, so, by the inductive hypothesis, $x_1^i, \ldots, x_k^i \vdash \lnot X$.
Moreover, $v_i(X) = 0$ implies that $v_i(A) = v_i(\lnot X) = 1$, so $A^i$ will be $\lnot X$ and, therefore, we can conclude that $x_1^i,\ldots,x_k^i \vdash A^i$.

If $v_i(X) = 1$, then $X^i$ will be $X$ and we will have $v_i(A) = v_i(\lnot X) = 0$, so $A^i$ will be $\lnot\lnot X$.
In addition, by the inductive hypothesis, we know that $x_1^i,\ldots,x_k^i\vdash X$.
In conjunction with \ref{pcl}\ref{pcl2}, this shows that $x_1^i,\ldots,x_k^i \vdash \lnot\lnot X$ which translates into our desired $x_1^i,\ldots,x_k^i\vdash A^i$.

Let us now assume $A$ to be of the form $X \limplies Y$. We will consider three different subcases: one for $v_i(X) = 0$, one for $v_i(Y) = 1$ and one for $v_i(X) = 1$ and $v_i(Y) = 0$.

If $v_i(X) = 0$, then $X^i$ will be $\lnot X$ and we will obviously have $v_i(A) = v_i(X \limplies Y) = 1$, so $A^i$ will be $A$.
According to our inductive hypothesis, we know that $x_1^i,\ldots,x_k^i\vdash \lnot X$. Taking \ref{pcl}\ref{pcl1} into consideration, it follows that $x_1^i, \ldots,x_k^i \vdash (X\limplies Y)$, hence $x_1^i, \ldots,x_k^i \vdash A$ as we wanted to show.

If $v_i(Y) = 1$, $Y^i$ will be $Y$ and we will also have $v_i(A) = v_i(X\limplies Y) = 1$. In this scenario, our hypothesis states that $x_1^i, \ldots,x_k^i \vdash Y$, which, using \ref[fund]{axp1}, leads us to $x_1^i,\ldots,x_k^i \vdash X \limplies Y$.

Lastly, if $v_i(X) = 1$ and $v_i(Y) = 0$, we will have $v_i(A) = 0$, so $A^i$ will be $\lnot(X \limplies Y)$. By the inductive hypothesis,
\[ x_1^i,\ldots,x_k^i \vdash X,\qquad x_1^i,\ldots,x_k^i \vdash \lnot Y,\]
which, together with \ref{pcl}\ref{pcl3}, leads to $x_1^i,\ldots,x_k^i \vdash \lnot(X\limplies Y)$ and completes our proof.
\end{proof}

\begin{theorem}
The formal system $\mathsf{P}$ of propositional logic is semantically complete.
\label{psemcom}
\end{theorem}

\begin{proof}
Let $A$ be any tautology. We aim to prove that it is a theorem in $\mathsf{P}$.

According to our previous lemma, letting $x_1,\ldots,x_k$ be the only distinct propositional variables appearing in $A$, if we fix any interpretation $i$, we will have $x_1^i,\ldots,x_k^i \vdash A^i$. Nonetheless, as $A$ is a tautology, it will be true under any interpretation $i$ and, therefore, $A^i$ will be $A$.

We can pick any two interpretations $i$ and $j$ that make $x_k$ true and false respectively and that are equal in all the remaining variables, this is, that satisfy $v_i(x_k) = 1$, $v_j(x_j) = 0$ and $v_i(x_l) = v_j(x_l)$ if $k \neq l$.
According to our previous lemma, we will have
\[ x_1^i,\ldots,x_{k-1}^i,x_k \vdash A,\qquad
x_1^i,\ldots,x_{k-1}^i, \lnot x_k \vdash A,\]
where we have used the fact that $x_l^i$ is the same as $x_l^j$ if $l \neq k$.
Applying the deduction theorem, we are thus led to
\[ x_1^i,\ldots,x_{k-1}^i \vdash (x_k \limplies A),\qquad
x_1^i,\ldots,x_{k-1}^i \vdash (\lnot x_k \limplies A).\]
If we now consider \ref{pcl}\ref{pcl4}, it is immediate that
\[ x_1^i,\ldots,x_{k-1}^i \vdash A.\]
A recursive application of this reasoning eventually leads to $\vdash A$.
\end{proof}


\begin{lemma}
Let $\mathsf{H}$ be a consistent first-order system defined on a certain first-order language.
There exists a consistent extension of the set of non-logical axioms of $\mathsf{H}$ (also known as an extension of $\mathsf{H}$) that makes $\mathsf{H}$ syntactically complete.
\label{fol-scext}
\end{lemma}

\begin{proof}
Let us consider an infinite enumeration $A_1,\ldots,A_n,\ldots$ of all the formulas in the first-order language under consideration.
Such an enumeration can indeed be constructed and I will leave the details for you.
We define a sequence of extensions of $\mathsf{H}$ as follows. Firstly, the formal system $\mathsf{H}_0$ will be $\mathsf{H}$ itself. Then, for every natural $n$, we will define $\mathsf{H}_n$ to be the extension of $\mathsf{H}_{n-1}$ including $\lnot A_n$ as an axiom if $A_{n}$ is not a theorem in $\mathsf{H}_{n-1}$. If it is a theorem, we define $\mathsf{H}_n$ to be $\mathsf{H}_{n-1}$.

By \ref[fund]{notacons}, we know that each of this extensions will be consistent. Thus, if we consider the extension $\mathsf{H}_{\infty}$ including as axioms all the axioms of all the $\mathsf{H}_n$ for every $n\in\mathbb{N}$, it is immediate that it will be consistent too.
In addition, it can be easily seen that $\mathsf{H}_\infty$ will be a syntactically complete extension. Just assume that there is a sentence $A$ such that neither $A$ or $\lnot A$ are theorems in $\mathsf{H}_\infty$. Shouldn't one of them have been added as an axiom in a certain  $\mathsf{H}_n$?
\end{proof}

\begin{theorem}
Any consistent first-order formal system $\mathsf{H}$ has a (countable!) model.
\label{foconmodel}
\end{theorem}

\begin{proof}
Let us consider the formal system $\mathsf{H}_0$ obtained by adding an infinite sequence of symbols $b_0,\ldots,b_n,\ldots$ to the particular language it uses.

It is easy to see that the addition of these symbols has no effect in the consistency of the system.
It suffices to show how a proof in the new formal system of any statement $A$ can be transformed into one, in the old system, of the formula $A'$ obtained by replacing every constant $b_1,\ldots,b_n,\ldots$ in $A$ by a new variable not occurring in $A$.
Thus, should the new system be able to prove both $A$ and $\lnot A$, the old one would prove $A'$ and $\lnot A'$, but that would be impossible for we have assumed $\mathsf{H}$ to be consistent.

We can now consider an enumeration $A_1,\ldots,A_n,\ldots$ of all the formulas in the language of $\mathsf{H}_0$ with a single free variable --- which we will denote, for each $A_n$, as $y_n$. 
From this point, we define a sequence of extensions of $\mathsf{H}_0$ as follows.
The first element of this sequence will obviously be $\mathsf{H}_0$.
Then, for every natural $n$, we will take $\mathsf{H}_n$ to be the extension of $\mathsf{H}_{n-1}$ incorporating the additional axiom $S_n$ given by
\[ A_n(y_n\Vert c_n) \limplies (\forall y_n) A_n,\]
where $c_n$ is the first element in $b_1,\ldots,b_n,\ldots$ that does not appear in $A$ and that does not belong to $\{c_1,\ldots,c_{n-1}\}$. 

Let us show that all these extensions will be consistent. Given any natural $n$, we will assume that $\mathsf{H}_n$ is not consistent and, therefore, that it can prove, for a certain formula $A$, both $A$ and $\lnot A$.
According to the principle of explosion, this means that we will be able to prove $\vdash_{\mathsf{H}_n} \lnot S_n$. Proofs in $\mathsf{H}_n$ are deductions from $S_n$ in $\mathsf{H}_{n-1}$. Considering this fact together with the deduction theorem yields
\[ \vdash_{\mathsf{H}_{n-1}} S_n \limplies \lnot S_n.\]
Since $(A \limplies \lnot A) \limplies \lnot A$ is a tautology, we can then deduce that $\lnot S_n$ will be a theorem in $\mathsf{H}_{n-1}$, so we will have
\[ \vdash_{\mathsf{H}_{n-1}} \lnot \left( A_n(y_n\Vert c_n) \limplies (\forall y_n) A \right).\]
As both $\lnot(A\limplies B) \limplies A$ and $\lnot(A \limplies B) \limplies \lnot B$ are tautologies, this means that
\[ \vdash_{\mathsf{H}_{n-1}} A_n(y_n\Vert c_n),\qquad
\vdash_{\mathsf{H}_{n-1}} \lnot (\forall y_n) A_n.\]
As $c_n$ does not appear in the axioms of $\mathsf{H}_{n-1}$, we can safely replace each occurrence of $c_n$ in the proof of $A_n(y_n\Vert c_n)$ by a variable not appearing in the proof.
Then, an application of the generalisation rule reveals that $\vdash_{\mathsf{H}_{n-1}} (\forall y_n) A_n$, which would mean that $\mathsf{H}_{n-1}$ would not be consistent, thus contradicting our hypothesis and showing the consistency of all the extensions in the sequence.

As all these extensions are consistent, the formal system $\mathsf{H}_\infty$ defined as the extension of $\mathsf{H}_0$ containing all the axioms $A_n$ will be consistent too.
We can now consider the (consistent) extension $\mathsf{H}^*$ of $\mathsf{H}_\infty$ obtained from a direct application of lemma \ref{fol-scext}.

We will now show that every element in the domain of any model of $\mathsf{H}^*$ can be referenced directly from the language.
Formally, this means showing that, given any formula $A_n$ with a single free variable $y_n$, if --- for every \emph{closed term} $t$ (term with no free variables) --- $A(y_n\Vert t)$ is a theorem, then so is $(\forall y_n) A$.

This is fairly easy to do. If we assume $A_n(y_n\Vert t)$ to be a theorem for every closed term $t$, then, in particular, $A(y_n\Vert c_n)$ will be a theorem. Thus, applying modus ponens on $S_n$, one can prove $\vdash_{\mathsf{H}_\infty} (\forall y_n) A$.

We will now define an interpretation of $\mathsf{H}^*$ that will be a model for it and, therefore, will also define a model for $\mathsf{H}$. This interpretation will have as domain of discourse the set of closed terms of the language of $\mathsf{H}^*$ (recall that it is the extension of the language of $\mathsf{H}$ with the additional constants $b_i$).
The interpretation of each constant $b_i$ will be itself and the functions will behave in the obvious way.
The interpretation of each $n$-ary predicate symbol $P$ will be the predicate that, for any closed terms $t_1,\ldots,t_n$ is true if and only if $P(t_1,\ldots,t_n)$ is a theorem in $\mathsf{H}^*$. 

We can then prove by induction on the number of connectives that any closed formula is a theorem in $\mathsf{H}^*$ if and only if it is true in the interpretation we have defined.
Taking the generalisation rule into consideration, this can be trivially extended to non-closed formulas, which implies that our interpretation is, indeed, a model of $\mathsf{H}^*$.
From the existence of a model of $\mathsf{H}^*$, the existence of a model for $\mathsf{H}$ can be trivially inferred.

Proving, by induction on the number of connectives, that closed formulas in $\mathsf{H}^*$ are theorems if and only if they are true is fairly straightforward and somewhat entertaining if you have a good understanding of the material. I will leave it to you.
Keep in mind that --- at least for the purpose of convincing yourself this result is true --- you don't need to write a fully-fledged formal proof.
A few diagrams and notes and a good dose of thinking will probably suffice.
\end{proof}
