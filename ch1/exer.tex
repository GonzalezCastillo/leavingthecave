%begin-include

\unsection{Exercises}

\begin{exercises}

\item The inverse image of functions has some truly nice and lovely properties: when working with functions and set operations, the inverse image is the mathematical equivalent of the love of your life. To see what I mean, let $f:X\longrightarrow Y$ and let $A,B\subseteq X$. Prove that
\begin{gather*}
f^{-1}[A\cup B] = f^{-1}[A] \cup f^{-1}[B],\\
f^{-1}[A\cap B] = f^{-1}[A] \cap f^{-1}[B],\\
f^{-1}[Y\setminus A] = X\setminus f^{-1}[A].
\end{gather*}
Can you deduce similar properties for the image of $f$?\footnote{That was a rhetorical question; of course you can. Please, do it.}


\item Find a counter-example for the converses of the statements in \ref{compinjsur}.

\item Let us regard $\iff$ as a relation between formulas of a first-order formal system: the relation $\iff$ holds between two formulas $A$ and $B$ if and only if $A\iff B$.
Prove that this relation is reflexive, symmetric and transitive.

Analogously, show that the relation induced by $\implies $ is transitive and reflexive but not necessarily symmetric.


\item Let $P$ be a formula in a first-order language.
Prove that, if $(\exists x) P$ is true in an interpretation, there exists an assignment of values $\alpha$ in that interpretation such that $v_\alpha(P) = 1$.

\item Let $A_1,\ldots,A_m$ and $B_1,\ldots,B_n$ be propositional forms for $m,n\in\mathbb{N}$. Prove that
\[ \vdash (A_1\land \cdots \land A_m) \limplies (B_1\land \ldots \land B_m)\]
if and only if, for every $i\in \{1,\ldots,n\}$, $A_1,\ldots,A_m\vdash B_i$.

\item The following statement is an alternative formulation of the axiom of choice: for every collection of sets $X$, there exists a \emph{choice function}
\[f: X\longrightarrow \bigcup X\]
assigning, to every $x\in X$, an element $a\in x$. Prove that this formulation is equivalent to the one we have used.

\item Show that the equality relation in any formal system using logic with equality verifies reflexivity, symmetry and transitivity.

\item Show that the axiom schema \ref{axp3} is independent in the formal system of propositional logic.

\item This exercise is a constructivist's nightmare.
Use the axiom of choice to prove the existence of a sequence of natural numbers \emph{without constructing a sequence}.
\end{exercises}
